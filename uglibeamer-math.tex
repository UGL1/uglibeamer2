\newcommand\barmin[1]{\overline{#1\vphantom{b}}\,}
\newcommand\barmaj[1]{\overline{#1\vphantom{X}}}
\newcommand\card{\textrm{Card}}


\newenvironment{pmatrice}															% Environnement pmatrice pour des écritures serrées
	{\renewcommand\arraystretch{0.7}\begin{pmatrix}}
	{\end{pmatrix}\renewcommand\arraystretch{1}}
\newcommand\repaff{\left(\ O \ ; \ I\ ; \ \ J\ \right)}								% Point et coordonnées dans le plan
\newcommand\rep{\left(O \ ; \ \vv{\imath}, \ \vv{\jmath}\right)}					% Repère du plan
\newcommand\repe{\left(O \ ; \ \vv{\imath}, \ \vv{\jmath}, \ \vv{k} \right)}		% Repère de l'espace

\newcommand\pc[3]{#1\left( #2 \ ; \ #3 \right)}										% Point et coordonnées dans le plan
\newcommand\pcf[1]{\pc{#1}{x_{#1}}{y_{#1}}}											% Point du plan et ses coordonnées dépendant du nom.
\newcommand\pce[4]{\textrm{#1}\left( #2 \ ; \ #3 \ ; \ #4 \right)}					% Idem dans l'espace
\newcommand\pcfe[1]{\pc{#1}{x_{#1}}{y_{#1}}{z_{#1}}}								% Idem

\newcommand\anglo[2]{\left( \vv{#1}  , \ \vv{#2} \right)}							% Angle orienté
\newcommand\vc[3]{\vv{#1}\begin{pmatrice} #2 \\ #3 \end{pmatrice}}					% Vecteur du plan et coordonnées
\newcommand\vce[4]{\vv{#1}\begin{pmatrice} #2 \\ #3 \\ #4 \end{pmatrice}}			% Vecteur de l'espace et coordonnées
\newcommand\cc[2]{\begin{pmatrice} #1 \\ #2 \end{pmatrice}}							% Coordonnées colonne
\newcommand\cce[3]{\begin{pmatrice} #1 \\ #2 \\ #3 \end{pmatrice}}					% Idem espace
\newcommand\norme[1]{\| \vv{#1} \|}													% Norme
\newcommand\ps[2]{\ensuremath{\vv{#1}\cdot\vv{#2}}}									% Produit scalaire de deux vecteurs du plan

\newcommand\N{\ensuremath{\mathbf{N}}} 												% Naturels
\newcommand\Z{\ensuremath{\mathbf{Z}}} 												% Relatifs
%\newcommand\D{\ensuremath{\mathbf{D}}} 												% Décimaux
\newcommand\Q{\ensuremath{\mathbf{Q}}}  												% Rationnels
\newcommand\R{\ensuremath{\mathbf{R}}} 												% Réels


\newcommand\oio[2]{\left]#1\ ;\ #2\right[}											% Intervalle borné ouvert
\newcommand\oif[2]{\left]#1\ ;\ #2\right]}											% Intervalle borné semi-ouvert semi-fermé
\newcommand\fio[2]{\left[#1\ ;\ #2\right[}											% Intervalle borné semi-fermé semi ouvert
\newcommand\fif[2]{\left[#1\ ;\ #2\right]}											% Intervalle borné fermé
\newcommand\iif[1]{\left]-\infty\ ;\ #1\right]}										% Coupure inférieure fermée
\newcommand\iio[1]{\left]-\infty\ ;\ #1\right[}										% Coupure inférieure ouverte
\newcommand\fii[1]{\left[#1\ ;\ +\infty\right[}										% Coupure supérieure fermée
\newcommand\oii[1]{\left]#1\ ;\ +\infty\right[}										% Coupure supérieure ouverte

\newcommand{\e}{\textrm{e}}															% exponentielle
\newcommand\courbe[1]{\ensuremath{\mathcal{C}_{#1}}}								% C_f
                                    % TES

